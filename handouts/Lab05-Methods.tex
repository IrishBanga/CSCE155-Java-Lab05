\documentclass[12pt]{scrartcl}



\setlength{\parindent}{0pt}
\setlength{\parskip}{.25cm}

\usepackage{graphicx}

\usepackage{xcolor}

\definecolor{darkred}{rgb}{0.5,0,0}
\definecolor{darkgreen}{rgb}{0,0.5,0}
\usepackage{hyperref}
\hypersetup{
  letterpaper,
  colorlinks,
  linkcolor=red,
  citecolor=darkgreen,
  menucolor=darkred,
  urlcolor=blue,
  pdfpagemode=none,
  pdftitle={CS1 - Lab 5.0 - Java},
  pdfauthor={Christopher M. Bourke},
  pdfkeywords={}
}

\definecolor{MyDarkBlue}{rgb}{0,0.08,0.45}
\definecolor{MyDarkRed}{rgb}{0.45,0.08,0}
\definecolor{MyDarkGreen}{rgb}{0.08,0.45,0.08}

\definecolor{mintedBackground}{rgb}{0.95,0.95,0.95}
\definecolor{mintedInlineBackground}{rgb}{.90,.90,1}

%\usepackage{newfloat}
\usepackage[newfloat=true]{minted}
\setminted{mathescape,
               linenos,
               autogobble,
               frame=none,
               framesep=2mm,
               framerule=0.4pt,
               %label=foo,
               xleftmargin=2em,
               xrightmargin=0em,
               startinline=true,  %PHP only, allow it to omit the PHP Tags *** with this option, variables using dollar sign in comments are treated as latex math
               numbersep=10pt, %gap between line numbers and start of line
               style=default, %syntax highlighting style, default is "default"
               			    %gallery: http://help.farbox.com/pygments.html
			    	    %list available: pygmentize -L styles
               bgcolor=mintedBackground} %prevents breaking across pages
               
\setmintedinline{bgcolor={mintedBackground}}
\setminted[text]{bgcolor={mintedBackground},linenos=false,autogobble,xleftmargin=1em}
%\setminted[php]{bgcolor=mintedBackgroundPHP} %startinline=True}
\SetupFloatingEnvironment{listing}{name=Code Sample}
\SetupFloatingEnvironment{listing}{listname=List of Code Samples}

\title{CSCE 155 - Java}
\subtitle{Lab 5.0 - Methods}
\author{~}
\date{~}

\begin{document}

\maketitle

\section*{Prior to Lab}

Before attending this lab:
\begin{enumerate}
  \item Read and familiarize yourself with this handout.
  \item Review the lecture notes on conditionals, or the following free textbook resource: \url{http://www.toves.org/books/java/ch12-methods/index.html}
\end{enumerate}

\section*{Peer Programming Pair-Up}

To encourage collaboration and a team environment, labs will be
structured in a \emph{pair programming} setup.  At the start of
each lab, you will be randomly paired up with another student 
(conflicts such as absences will be dealt with by the lab instructor).
One of you will be designated the \emph{driver} and the other
the \emph{navigator}.  

The navigator will be responsible for reading the instructions and
telling the driver what to do next.  The driver will be in charge of the
keyboard and workstation.  Both driver and navigator are responsible
for suggesting fixes and solutions together.  Neither the navigator
nor the driver is ``in charge.''  Beyond your immediate pairing, you
are encouraged to help and interact and with other pairs in the lab.

Each week you should alternate: if you were a driver last week, 
be a navigator next, etc.  Resolve any issues (you were both drivers
last week) within your pair.  Ask the lab instructor to resolve issues
only when you cannot come to a consensus.  

Because of the peer programming setup of labs, it is absolutely 
essential that you complete any pre-lab activities and familiarize
yourself with the handouts prior to coming to lab.  Failure to do
so will negatively impact your ability to collaborate and work with 
others which may mean that you will not be able to complete the
lab.  

\section{Lab Objectives \& Topics}
At the end of this lab you should be familiar with the following
\begin{itemize}
  \item Know how to write and use methods
  \item Know the difference between a method prototype and method definition
\end{itemize}

\section{Background}

Most programming languages allow you to define and use functions 
(or methods).  Functions are reusable blocks of code that can be 
packaged as a single unit.  A function can be specified to inputs 
(parameters, arguments) and return an output.  Defining functions 
has several advantages.  First, it facilitates code reuse.  Rather than 
cutting and pasting the same block of code, it can be encapsulated 
into a function and reused by calling the function anytime it needs 
to be executed.

Second, functions facilitate \emph{procedural abstraction}.  Often 
times, we don't care or need to worry about the implementation 
details of a certain algorithm or procedure.  By encapsulating the 
details in a function, we only need to know how to use it (what 
inputs to provide it and what output we can expect from it) and 
not need to worry about how it computes its result.  For example,
up to now you've been using the standard math library's function
to compute the square root of a number $x$, but you haven't
had to worry about the details of how this computation actually
takes place.

When defining a function, it is necessary to declare its \emph{signature}.  
The signature of a function includes:
\begin{itemize}
  \item The function's identifier -- its name 
  \item The return type -- the type of variable the function returns)
  \item The parameter list -- the number of parameters the function takes 
	(also called its \emph{arity}) along with the types 
\end{itemize}

\subsection{Methods in Java}

In Java, functions (usually called methods) must be declared/defined 
within a class.  This is done by declaring the method?s signature and 
adding a block of code that specifies the instructions that will be executed 
when the method is invoked.  In addition, there are two modifiers that 
can be applied to methods:
\begin{itemize}
  \item \mintinline{java}{public} -- this specifies that the method is publicly 
  	visible and can be invoked by any other method.  Alternatives include 
	\mintinline{java}{private} (the method is only visible within the class) 
	and \mintinline{java}{protected} (the method is visible within the class 
	and any subclasses)
  \item \mintinline{java}{static} -- this specifies that the method belongs to the 
	class and not to instances of the class.
\end{itemize}

\section{Activities}

We have provided partially completed programs for each of the 
following activities.  Clone the lab's code from GitHub using the 
following URL: \url{https://github.com/cbourke/CSCE155-Java-Lab05}.

\subsection{Using Methods}

The file \mintinline{text}{OrderStatistic.java} contains code necessary
to find the $i$-th order statistic of a collection of numbers.   The $i$-th order 
statistic corresponds to the $i$-th element in a sorted list.  For example, 
the 4th order statistic of the list [5, 99, 23, 14, 6] is 23; the sorted list 
is [5, 6, 14, 23, 99], and 23 is the 4th element.  Some special cases:
\begin{itemize}
  \item The 1-th order statistic is the \emph{minimum} element
  \item The $n$-th order statistic is the \emph{maximum} element
  \item The $\frac{n}{2}$-th order statistic is the \emph{median} element
\end{itemize}
The program converts command line arguments into $i$ and an
array of integers.  It then sorts the array and outputs the $i$-th 
element.  It does so by calling a series of functions.

\subsubsection*{Instructions}

\begin{enumerate}
  \item Complete question 1 on your worksheet.
  \item Open the \mintinline{text}{OrderStatistic.java} source file and
	study the code (do not make any changes).  In particular, read
	the signatures for each method and their documentation to
	learn what each one does.  
  \item Compile and run the program with the following input values: \\
	\mintinline{text}{4 99 23 76 100 8 3 0 1 72 104 1000 12 18 14}
  \item Answer the relevant questions on your worksheet.
\end{enumerate}

\subsection{Writing A Method}

Open the \mintinline{text}{Sine.java} source file.  This is a skeleton 
program similar to a previous lab exercise that uses a Taylor series
to compute $\sin{x}$.  Modify the program so that the code that
computes $\sin{x}$ is encapsulated into a function:

\begin{enumerate}
  \item Write a method signature for this method: what is its return
  	type?  What are its parameters?  
  \item Write the method definition by moving the code from the \mintinline{java}{main}
  	method to your method.
  \item Modify the code in the \mintinline{java}{main} method to use your new
  	method.
  \item Compile and test your program.
\end{enumerate}

\subsection{Writing More Methods}

Open the \mintinline{text}{PhysicsCalc.java} source file.  This is a skeleton 
program that you'll complete using your own methods.  This program 
will prompt the user for the following input:
\begin{itemize}
  \item The distance an object traveled from point 1 to point 2 (variable distance)
  \item The time required to travel distance (variable time)
  \item The object's initial velocity (variable initVelocity)
  \item The mass of the object (variable mass)
\end{itemize}

You will need to write a series of methods that will calculate the force of the 
object using the instructions below.  Each method should be independent 
from the others (i.e., the force method shouldn't call the acceleration method, 
and the acceleration method shouldn't call the velocity method).  

\begin{enumerate}
  \item Write a method that calculates the average velocity, using the equation 
	$$v = \frac{\textrm{distance}}{\textrm{time}}$$
  \item Write a method that calculates the average acceleration using the equation
	$$a = \frac{v - v_0}{\textrm{time}}$$
  	where $v$ is the current velocity and $v_0$ is the initial velocity.  For your 
	program, use the average velocity as the current velocity.
  \item Write another method that calculates the force of an object using the equation
	$$F = ma$$
	where $m$ is the mass of the object and $a$ is the acceleration.  
  \item Add a print statement to print your result to the screen.
  \item Answer question 3 on the worksheet and demonstrate your program to 
  	the lab instructor.
\end{enumerate}
	
\section{Advanced Activity (Optional)}

Write a few more methods in the \mintinline{text}{OrderStatistics.java} program.  First,
write a method that gets the \emph{median} of a list of elements.  However, the
method should \emph{not} make changes to the array.  Instead, write additional
methods that create a \emph{copy} of the array and sorts the copy so that it can
find the median element.  

\end{document}
